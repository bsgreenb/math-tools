\documentclass[11pt]{article}
\usepackage{math-notes-preamble}

\title{DiffEq}

\begin{document}
\maketitle

\section{Slope fields}\label{slope-fields}

\href{https://youtu.be/8Amgakx5aII}{Vid on Khan}

\section{Euler's Method}\label{eulers-method}

\href{https://youtu.be/q87L9R9v274}{Vid on Khan}

\section{Separable Differential
Equations}\label{separable-differential-equations}

Take the form

\[
\frac{dy}{dx} = g(x)h(y)
\]

\href{https://youtu.be/DL-ozRGDlkY}{Vid on Khan}. Easiest type of one to
solve without needing numerical approximation methods.

These are differential equations where you can get y and dy on one side,
and x and dx on the other side. Then you can integrate both sides.

Note that this involves doing mathematically meaningless stuff with
floating dys and dxs.
\href{https://math.stackexchange.com/a/1252426/49487}{Explained here on
MSE} why this works.

{[}add formula from wiki{]}

Note: Be extra careful to keep track of Constants of Integration. They
can determine the answer.

\section{Growth Models}\label{growth-models}

\href{https://youtu.be/_JpS8k1a9yE}{Khan Vid} covers the way
\(N = Ce^{rt}\) from \(\frac{dN}{dt} = rN\). Then Verhulst gave a
rigorous version of Malthus which incorporated restraints which is

\section{Logistic Differential
Equation}\label{logistic-differential-equation}

On AP Calc \href{https://youtu.be/NU1v-8VRirU}{Logistic Equations}, we
start with the Logistic Differential Equation:

\[\frac{dN}{dt} = rN(1-\frac{N}{k})\]

And we want to get to \(N(t)\). This involves algebraically rearrange
the differntial equation and taking advantage of partial fractions to
get it more integrable on the left hand side, til we arrive at:

\[
\frac{N}{1-\frac{N}{k}} = Ce^{rt} 
\]

Now our best bet is to take the reciprocal {[}Is this the general
Bernoulli Diff Eq approach?{]}

\[
\frac{1-\frac{N}{k}}{N} = Ce^{-rt} \\
\frac{1}{N} = Ce^{-rt} +\frac{1}{k} \\
N(t) = \frac{1}{Ce^{-rt} +\frac{1}{k}} \\
\]

Now we want to solve for \(C\) by looking at \(N(0)\)

\[
N_0 = \frac{1}{C +\frac{1}{k}} \\
\frac{1}{N_0} = C + \frac{1}{k} \\
C = \frac{1}{N_0} - \frac{1}{k} \\
\]

Plugging C back in:

\[
N(t) = \frac{1}{(\frac{1}{N_0} - \frac{1}{k})e^{-rt} +\frac{1}{k}} \\
\]

Let's clean up those fractions in the denominator by multiplying: \[
N(t) = \left(\frac{1}{(\frac{1}{N_0} - \frac{1}{k})e^{-rt} +\frac{1}{k}}\right)\left(\frac{N_0k}{N_0k}\right) \\
\]

\[
N(t) = \frac{N_0k}{(k - N_0)e^{-rt} + N_0} \\
\]

\subsection{Inflection Point of Logistic
Curve}\label{inflection-point-of-logistic-curve}

Just like any other function, find when \(f'' = 0\).

\section{Bernoulli Differential
Equation}\label{bernoulli-differential-equation}

Logistic Differential Equation is an example of a
\href{https://brilliant.org/wiki/bernoullis-equation/}{Bernoulli
Differential Equation}.

\end{document}
