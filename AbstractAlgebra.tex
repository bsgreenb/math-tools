\documentclass[11pt]{article}
\usepackage{math-notes-preamble}

\title{Abstract Algebra}

\begin{document}
\maketitle

Notes here are based on Pinter - A Book of Abstract Algebra, 2nd
Edition. I focused on just definitions and key theorems, unlike my
Linear Algebra notes where I feel like I way overdid it.

\section{Algebras}\label{algebras}

Every \textbf{algebra} consists of a set and certain operations on that
set.

An \textbf{algebraic structure} is a set with one or more operations
defined on it.

\section{Operations}\label{operations}

An \textbf{operation} \(\circ\) on \(A\) is a rule which assigns to each
ordered pair \((a, b)\) of elements of A exactly one element
\(a \circ b\) in \(A\).

\subsection{Algebraic Properties}\label{algebraic-properties}

Properties of operations on a set \(A\):

\begin{itemize}
\tightlist
\item
  \textbf{Closure}: The operation on elements in \(A\) always produces
  another element of \(A\).
\item
  \textbf{Commutativity}: $a \circ b = b \circ a$ for any two
  elements \(a\) and \(b\) in \(A\).
\item
  \textbf{Associativity}: \((a \circ b) \circ c = a \circ (b \circ c)\)
  for any three elements \(a,b,c\) in \(A\).
\end{itemize}

Other algebraic properties include existence of an identity/neutral
element, existence of an inverses of all elements, and distributivity
(which applies to structures with multiple operations). More on all
these to follow.

\subsection{Neutral Element}\label{neutral-element}

If there exists an element \(e\) in \(A\) such that

\(e \circ a = a\) and \(a \circ e = a\) for every \(a\) in \(A\)

we call \(e\) an \textbf{identity element} aka \textbf{neutral element}
for the operation \(\circ\).

0 is identity element for addition, 1 for multiplication.

\subsection{Inverses}\label{inverses}

An \textbf{inverse} \(a^{-1}\) is defined by causing \(a\) to reach the
neutral element \(e\):

\(a \circ a^{-1} = e\) and \(a^{-1} \circ a = e\)

    \section{Groups}\label{groups}

A \textbf{group} is a set \(G\) with an operation \(\circ\) which
satisfies the axioms: 1. \(\circ\) is associative. 2. There is an
element \(e\) in \(G\) such that \(a \circ e = a\) and \(e \circ a = a\)
for every element \(a\) in \(G\). 3. For every element \(a\) in \(G\),
there is an element \(a^{−1}\) in \(G\) such that \(a \circ a^{-1} = e\)
and \(a^{-1} \circ a = e\).

The notation for the above group would be \(\left<G, \circ\right>\).

Note that the definition of an operation ensures that we always stay
within \(G\) when performing the operation.

If the commutative law holds in a group \(G\), such a group is called an
\textbf{abelian group} or a \textbf{commutative group}.

The number of elements in a finite group \(G\) is called its
\textbf{order}. It's denoted by \(|G|\).

We say that two elements in a group \textbf{commute} if \(ab = ba\).

The \textbf{direct product} of a group
\(G \times H = \{(x, y) : x \in G \land y \in H\}\)

Every finite group can be represented by a \textbf{Cayley Diagram}. Each
point is an element of the group, and the directional arrows indicate
the result of multiplying by their generators. Different line colors or
dottings can be used to indicate which generator an arrow refers to.
There are no arrows when there's bidirectionality because a given
generator is its own inverse.

\section{Subgroups}\label{subgroups}

Let \(G\) be a group, and \(S\) a nonempty subset of \(G\). If the
product of every pair of elements of \(S\) is in \(S\), we say that
\(S\) is \textbf{closed with respect to multiplication}. If the inverse
of every element of \(S\) is in \(S\), we say that \(S\) is
\textbf{closed with respect to inverses}. If both of these are true, we
call \(S\) a \textbf{subgroup} of \(G\) (and itself a group).

In any group \(G\) the one-element subset \(\{e\}\), containing only the
neutral element, is a subgroup. At the other extreme, the whole group
\(G\) is obviously a subgroup of itself. These two examples are,
respectively, the smallest and largest possible subgroups of \(G\). They
are called the \textbf{trivial subgroups} of \(G\). All the other
subgroups of \(G\) are called \textbf{proper subgroups}.

A \textbf{generator} of a group \(G\) is a subset \(S \subseteq G\) such
that every element of \(G\) can be expressed as a finite product of
elements from \(S\) and their inverses.

The \textbf{subgroup generated by} a subset \(S\), denoted
\(\langle S \rangle\), of a group \(G\) is the smallest subgroup of
\(G\) containing \(S\), equivalently the set of all finite products of
elements of \(S\) and their inverses.

A concise definition of the subgroup generated by \(a\) could be given
by:

\[
\langle a\rangle:=\left\{a^k \mid k \in \mathbb{Z}\right\}
\]

A set of equations, involving only the generators (which includes their
inverses), is called a set of \textbf{defining equations} for \(G\) if
these equations completely determine the table of \(G\).

By the \textbf{center} of a group \(G\) we mean the set of all the
elements of \(G\) which commute with every element of \(G\), that is,
\[C = \{a \in G : ax = xa \text{ for every } x \in G\}\]

\section{Groups of Permutations}\label{groups-of-permutations}

A \textbf{permutation} of a set \(A\) is a bijection
\(A \rightarrow A\).

Because composite bijections are themselves a bijection, the composite
of any two permutations is a permutation.

The set of all permutations of \(A\) with the operation \(\circ\) of
composition, is a group. The group of all permutations for a given set
\(A\) is called the \textbf{symmetric group} on \(A\). The symmetric
group on \(n\) elements is denoted by \(S_n\).

The \textbf{dihedral group}, denoted \(D_n\) corresponds to the group of
symmetries of an \(n\)-gon. These symmetries consist of rotation and
reflection subgroups.

See
\href{https://proofwiki.org/wiki/Definition:Symmetry_Group_of_Square}{the
symmetries of a square}. Rotations are like if you stuck a pole through
it like a pig on a spit.

\section{Permutations of a Finite
Set}\label{permutations-of-a-finite-set}

Let \(a_1, a_2, \ldots, a_s\) be distinct elements of the set
\(\{1,2,\ldots, n\}\). By the \textbf{cycle} \((a_1a_2 \ldots a_s)\) we
mean the permutation of \(\{1,2,\ldots, n\}\) which carries \(a_1\) to
\(a_2\), \(a_2\) to \(a_3\),\(\ldots\), \(a_{s-1}\) to \(a_s\), and
\(a_s\) to \(a_1\), while leaving all the remaining elements of
\(\{1,2,\ldots,n\}\) fixed. (Note the conection with modular addition: a
cycle maps \(a_i\) \(a_{i+1 \mod s}\)

Because cycles are permutations, we may form the \textbf{composite} of
two cycles in the usual manner. The composite of cycles is generally
called their \textbf{product} and it is customary to omit the symbol
\(\circ\).

If \((a_1a_2 \ldots a_s)\) is a cycle, the integer \(s\) is called its
\textbf{length}; thus, \((a_1a_2 \ldots a_s)\) is a cycle of length
\(s\).

If two cycles have no elements in common they are said to be
\textbf{disjoint}. Disjoint cycles commute.

\textbf{Theorem}: Every permutation is either the identity, a single
cycle, or a product of disjoint cycles (unique up to reordering the
cycles and cyclic shifts within cycles).

A cycle of length 2 is called a \textbf{transposition}. Every cycle can
be expressed as a product of one or more transpositions. In fact,

\[
\left(a_1 a_2 \ldots a_r\right)=\left(a_r a_{r-1}\right)\left(a_r a_{r-2}\right) \ldots\left(a_r a_3\right)\left(a_r a_2\right)\left(a_r a_1\right)
\]

However, the expression of a permutation as a product of transpositions
is not unique, and even the number of transpositions involved is not
unique. Nevertheless, when a permutation is written as a product of
transpositions, one property of this expression is unique: the number of
transpositions involved is either always even or always odd. A
permutation is called \textbf{even} if it is a product of an even number
of transpositions, and \textbf{odd} if it is a product of an odd number
of transpositions.

\textbf{Theorem}: The identity permutation is even.

When you take the product of two permutations, the oddness or evennness
of the composite permutation follows the same rule as odd/even for
addition. Two odds or evens compose to an even, and one odd and one even
compose to an odd.

The set of all the even permutations in \(S_n\) is a subgroup of
\(S_n\). It is called the \textbf{alternating group}, and is denoted by
\(A_n\).

If \(\alpha\) is any permutation, the least positive integer \(n\) such
that \(\alpha^n = \epsilon\) is called the \textbf{order} of \(\alpha\).

\subsection{Powers of Permutations}\label{powers-of-permutations}

Let \(\alpha\) be a cycle of length \(s\).

There will be \(s\) distinct powers of \(\alpha\). \(\alpha^{s}\) will
be the identity permutation. \(\alpha^{s-1}\) will be \(\alpha^{-1}\).

\(\alpha^2\) is a cycle iff \(s\) is odd. If \(s\) is odd, the square
root of \(\alpha\) will be \(\alpha^{\frac{s+1}{2}}\) (this follows from
the fact that \(\alpha = \alpha^{s+1}\)).

\textbf{Cycle Decomposition Theorem}: \(\alpha^k\) is the product of
\(d=\operatorname{gcd}(s, k)\) disjoint cycles of length
\(t=\frac{s}{d}\). Specifically:

\[
\alpha^k=\prod_{m=1}^d(m, m+k, m+2 k, \ldots, m+(t-1) k)
\]

From this we can see that when \(s\) and \(k\) are coprime
(\(\operatorname{gcd}(s, k) = 1\)), \(\alpha^k\) will be a single cycle.
Furthmore, when \(s\) is prime, the only possible
\(\operatorname{gcd}(s, k)\) is either 1 (another single cycle) or
\(s\), in which case we have cycles of length 1 -- resulting in the
identity permutation.

\subsection{Conjugate Cycles}\label{conjugate-cycles}

If \(\alpha\) is any cycle, and \(\pi\) any permutation,
\(\pi\alpha\pi^{-1}\) is called the \textbf{conjugate}.

The conjugate of \(\alpha\) is the cycle which ``labels'' the cycle,
\((\pi(\alpha_1),\ldots,\pi(\alpha_s))\)

When we say two cycles are conjugates of each other, we mean that there
exists a permutation such that one cycle can be transformed into the
other by relabeling its elements. Note that the inverse of that
permutation gets things to work in the opposite direction.

Any two cycles of equal length are conjugates of each other.

Let \(\sigma\) be a product \(\alpha_1 \ldots \alpha_t\) of \(t\)
disjoint cycles of lengths \(l_1 \ldots, l_t\), respectively. Then
\(\pi \sigma \pi^{-1}\) is also a product of \(t\) disjoint cycles of
lengths \(l_1, \ldots, l_t\). The proof for this involves the
observation that

\[
\pi \sigma \pi^{-1}=\pi a_1 a_2 \cdots a_t \pi^{-1}=\left(\pi a_1 \pi^{-1}\right)\left(\pi a_2 \pi^{-1}\right) \cdots\left(\pi a_t \pi^{-1}\right) .
\]

\subsection{Order of Cycles}\label{order-of-cycles}

If \(\alpha\) is any permutation, the least positive integer \(n\) such
that \(\alpha^n = \epsilon\) is called the order of \(\alpha\).

If \(\alpha\) is any cycle of length \(s\), the order of \(\alpha\) is
\(s\).

If \(\alpha\) and \(\beta\) are disjoint cycles of lengths \(r\) and
\(s\), respectively, the order of \(\alpha\beta\) is
\(\text{lcm}({\alpha\beta})\).

    \section{Isomorphism}\label{isomorphism}

Isomorphic things have the same structure.

Let \(G_1\) and \(G_2\) be groups. A bijective function
\(f: G_1 \rightarrow G_2\) with the property that for any two elements
\(a\) and \(b\) in \(G_1\),

\[f(ab) = f(a)f(b)\]

is called an \textbf{isormophism} from \(G_1\) to \(G_2\). If there
exists an isomorphism from \(G_1\) to \(G_2\), we say that \(G_1\) is
\textbf{isormophic} to \(G_2\).

Notation for congruence is \(G_1 \cong G_2\).

\textbf{Cayley's Theorem}: Every group is isomorphic to a group of
permutations. (Recall that a group of permutations is a subgroup of
\(S_n\).)

\subsection{Isomorphism Is an Equivalence Relation Among
Groups}\label{isomorphism-is-an-equivalence-relation-among-groups}

Every group is isomorphic to itself. If \(G_1 \cong G_2\), then
\(G_2 \cong G_1\). If \(G_1 \cong G_2\) and \(G_2 \cong G_3\), then
\(G_1 \cong G_3\).

The identity function is always an isomorphism from a group to itself.

\subsection{Elements Which Correspond under an
Isomorphism}\label{elements-which-correspond-under-an-isomorphism}

If we have an isomorphism \(f: G_1 \rightarrow G_2\):

\begin{itemize}
\tightlist
\item
  \(f\) matches the neutral element of \(G_1\) with the neutral element
  of \(G_2\).
\item
  If \(f\) matches an element \(x\) in \(G_1\) with \(y\) in \(G_2\),
  then \(f\) matches \(x^{-1}\) with \(y^{-1}\).
\item
  \(f\) matches a generator of \(G_1\) with a generator of \(G_2\).
\end{itemize}

\subsection{Some General Properties of
Isomorphism}\label{some-general-properties-of-isomorphism}

\begin{itemize}
\tightlist
\item
  \(G \times H \cong H \times G\)
\item
  If \(G_1 \cong G_2\) and \(H_1 \cong H_2\), then
  \(G_1 \times H_1 \cong G_2 \times H_2\)
\item
  \(G\) is abelian iff \(f(x) = x^{-1}\) is an isomorphism from \(G\) to
  \(G\).
\end{itemize}

\subsection{Group Automorphisms}\label{group-automorphisms}

If \(G\) is a group, an \textbf{automorphism} of \(G\) is an isomorphism
from \(G\) to \(G\). The identity function is always an automorphism of
a group, but there can be others beside this obvious one.

Since each automorphism of \(G\) is a bijective function from \(G\) to
\(G\), it is a permutation of \(G\).

\subsection{Regular Representation of
Groups}\label{regular-representation-of-groups}

By Cayley's theorem, every group \(G\) is isomorphic to a group \(G^*\)
of permutations of \(G\). Recall that we match each element \(a\) in
\(G\) with the permutation \(\pi_a = ax\), that is, ``multiply on the
left by \(a\).'' We let \(G^* = \{\pi_a : a \in G \}\); with the
operation \(\circ\) of composition it is a group of permutatiojns,
called the \textbf{left regular representation} of \(G\). It is called a
``representation'' of \(G\) because it is isomorphic to \(G\).

Instead of using the permutations \(\pi_a\), we could just as well have
used the permutations \(\rho_a\), defined by \(\rho_a(x) = xa\), that
is, ``multiply on the right by \(a\). The group
\(G^\rho = \{\rho_a : a \in G \}\) is called the \textbf{right regular
representation} of \(G\).

If \(G\) is commutative, there is no difference between the right and
left multiplication, so \(G^*\) and \(G^\rho\) are the same, and are
simply called the \textbf{regular representation} of \(G\).

    \section{Order of Group Elements}\label{order-of-group-elements}

Let \(G\) be a group, and \(a\) and element of \(G\). If there exists a
nonzero integer \(m\) such that \(a^m = e\), then there exists a
positive integer \(n\) such that \(a^n = e\). Specifically, if
\(m < 0\), then \(m = -n\) works because \(e^{-1} = e\).

If there exists a nonzero integer \(m\) such that \(a^m = e\), then the
\textbf{order} of the element \(a\) is defined to be the least positive
integer \(n\) such that \(a^n = e\). If there does not exist any nonzero
integer \(m\) such that \(a^m = e\), we say that \(a\) has order
infinity.

The finite order of an element \(g\) of a group can be defined as:

\[
\operatorname{ord}(g)=\min \left\{k \in \mathbb{N} \mid g^k=e\right\}
\]

Thus, in any group G, every element has an order which is either a
positive integer or infinity. If the order of \(a\) is a positive
integer, we say that \(a\) has \textbf{finite order}, otherwise \(a\)
has \textbf{infinite order}.

In the following theorems, \(G\) is an arbitrary group, and \(a\) is any
element of \(G\).

\textbf{Theorem}: if \(a\) has finite order \(n\), there are exactly
\(n\) different powers of \(a\), namely:

\[
a^0, a, a^2, a^3, \ldots, a^{n-1}
\]

What this theorem asserts is that every positive or negative power of
\(a\) is equal to one of the above, and the above are all different from
one another.

\textbf{Theorem}: If \(a\) has infinite order, then all the powers of
\(a\) are different. That is, if \(r\) and \(s\) are distinct integers,
then \(a^r \ne a^s\).

\textbf{Theorem}: Suppose an element \(a\) in a group has order \(n\).
Then \(a^t = e\) iff \(t\) is a multiple of \(n\).

\subsection{Elementary Properties of
Order}\label{elementary-properties-of-order}

Let \(a\), \(b\), and \(c\) be elements of a group \(G\). Then:

\(\operatorname{ord}(a) = \operatorname{ord}(bab^{1})\)

The order of \(a{^1}\) is the same as the order of \(a\).

The order of \(ab\) is the same as the order of \(ba\).

\(\operatorname{ord}(abc) = \operatorname{ord}(cab) = \operatorname{ord}(bca)\).

Let \(x=a^1 a^2 \cdots a^n\), and let \(y\) be a product of the same
factors, permuted cyclically. (That is,
\(y=a_k a_{k+1} \cdots a_n a_1 \cdots a_{k-1}\).) Then
\(\operatorname{ord}(x)=\operatorname{ord}(y)\).

\subsection{Further Properties of
Order}\label{further-properties-of-order}

If \(a^p=e\) where \(p\) is a prime number, then \(a\) has order \(p\).
\((a \neq e\)).

The order of \(a^k\) is a divisor (factor) of the order of \(a\).

If \(\operatorname{ord}(a)=k m\), then
\(\operatorname{ord}\left(a^k\right)=m\).

If \(\operatorname{ord}(a)=n\) where \(n\) is odd, then
\(\operatorname{ord}\left(a^2\right)=n\).

\subsection{Relationship between ord(ab), ord(a), and
ord(b)}\label{relationship-between-ordab-orda-and-ordb}

Let \(a\) and \(b\) be elements of a group \(G\). Let
\(\operatorname{ord}(a)=m\) and \(\operatorname{ord}(b)=n\); let
\(\operatorname{lcm}(m\), \(n\) ) denote the least common multiple of
\(m\) and \(n\).

If \(a\) and \(b\) commute, then \(\operatorname{ord}(a b)\) is a
divisor of \(\operatorname{lcm}(m, n)\).

If \(m\) and \(n\) are relatively prime, then no power of \(a\) can be
equal to any power of \(b\) (except for \(e\) ). (REMARK: Two integers
are said to be relatively prime if they have no common factors except ±
1.)

If \(m\) and \(n\) are relatively prime, then the products
\(a^i b^j(0 \leq i<m, 0 \leq j<n)\) are all distinct.

Let \(a\) and \(b\) commute. If \(m\) and \(n\) are relatively prime,
then \(\operatorname{ord}(a b)=m n\).

Let \(a\) and \(b\) commute. There is an element \(c\) in \(G\) whose
order is \(\operatorname{lcm}(m, n)\).

Thus, there is no simple relationship between
\(\operatorname{ord}(a b)\), \(\operatorname{ord}(a)\), and
\(\operatorname{ord}(b)\) if \(a\) and \(b\) fail to commute.

\subsection{Relationship between ord(a) and
ord(a\^{}k)}\label{relationship-between-orda-and-ordak}

\end{document}
