\documentclass[11pt]{article}
\usepackage{math-notes-preamble}

\title{MultiVariableCalc}

\begin{document}
\maketitle

A function is called \textbf{multivariable} if its input is made up of
multiple numbers, e.g.~\(f(x,y)\).

If the output consists of multiple numbers, it may also be called
multivariable, but these ones are typically called \textbf{vector-valued
functions}. An example would be
\(f(x) = \begin{bmatrix}\cos(x) \\ \sin(x)\end{bmatrix}\).

\section{Derivatives of multivariable
functions}\label{derivatives-of-multivariable-functions}

\subsection{Partial Derivatives}\label{partial-derivatives}

The partial derivative only cares about the changing of one of the
variables in a multivariable function. For example, the partial
derivative with respect to \(x\) of \(f(x,y)\) would be:
\(\frac{\partial f}{\partial x}\). Notationally, we can also represent a
partial derivative with respect to \(x\) as \(f_x\).

Solving these seems to work how you'd expect, the unchanging parameter
just acts as constants in your formula.

\subsubsection{Formal definition of partial
derivative}\label{formal-definition-of-partial-derivative}

Suppose we have \(f(x,y)\). Then

\[\frac{\partial f}{\partial x}(a, b) = \lim_{h \to 0}\frac{f(a + h, b) - f(a,b)}{h}\]

\subsubsection{Second partial
derivatives}\label{second-partial-derivatives}

We can use the notation \(\frac{\partial^2f}{\partial x^2}\) to indicate
the double partial derivative with respect to \(x\). To indicate the
derivative with respect to \(x\) then the derivative of that with
respect to \(y\), we would write:
\(\frac{\partial^2f}{\partial y \partial x}\). Sometimes you may see a
notation \(f_{xx}\) or \(f_{xy}\) respectively.

What we'll find is that there's often a symmetry between second partial
derivatives. When that symmetry is there, applying the partial
derivative with respect to \(y\) then with respect to \(x\) is
equivalent to doing partial with respect to \(x\) then with respect to
\(y\). In these cases
\({\partial^2f}{\partial y \partial x} = {\partial^2f}{\partial x \partial y}\).
The conditions for this symmetry existing are
\href{https://en.wikipedia.org/wiki/Symmetry_of_second_derivatives}{described
here on wiki}. Most of the time we'll find this to be true, the
exceptions occur when there's not continuity at the key point.

\subsubsection{Partial derivative of vector-valued
function}\label{partial-derivative-of-vector-valued-function}

Just take the partial derivatives of each of the components.

    \subsection{Gradient}\label{gradient}

The \textbf{gradient} is a way of packing together all the partial
derivative information of a function.

Say we have \(f(x,y) = x^2\sin(y)\). Then
\(\frac{\partial f}{\partial x} = 2x\sin(y)\), and
\(\frac{\partial f}{\partial y} = x^2\cos(y)\). The gradient is a vector
which has the partial derivatives in it,
\(\nabla f = \begin{bmatrix} 2x\sin(y) \\ x^2\cos(y)\end{bmatrix}\).

More broadly,
\(\nabla f(x,y) = \begin{bmatrix}\frac{\partial f}{\partial x} \\ \frac{\partial f}{\partial y}\end{bmatrix}\)

    \subsection{Directional Derivatives}\label{directional-derivatives}

With directional derivatives, we take an infinitesimal step in the
direction of a vector. The notation here is \(\nabla_{\vec{v}}f\), where
\(\vec{v}\) is the vector you're infinitesimely nudging in the direction
of.

We can solve for the directional derivative by taking the dot product of
the directional vector and the gradient:

\[
\vec{v} \cdot \nabla f
\]

Let's jump to the formal definition:

\[
\nabla_{\vec{v}}f(\vec{a}) = \lim_{h\to 0}\frac{f(\vec{a} + h\vec{v}) - f(\vec{a})}{h}
\]

\subsubsection{Directional derivatives and
slope}\label{directional-derivatives-and-slope}

Only if the directional vector is a unit vector, would you be able to
consider the directional derivative to be a slope. So a proper equation
for getting the slope regardless of the directional vector would be:

\[\frac{\vec{v} \cdot \nabla f}{||\vec{v}||}\]

    \subsection{Multivariable Chain Rule}\label{multivariable-chain-rule}

Simple version of the rule with two dimensional input:

\[\frac{d}{dt}f(x(t), y(t)) = \frac{\partial f}{\partial x}\frac{dx}{dt} + \frac{\partial f}{\partial y}\frac{dy}{dt}\]

More generally we can say:

\[\frac{d}{dt}f(g(t)) = \nabla f(g(t)) \cdot g'(t)\]

If we consider \(g(t)\) to be a vector-valued function \(\vec{v}(t)\),
We can compactly write this all in vector notation as:

\[\nabla f(\vec{v}(t)) \cdot \vec{v}'(t)\]

This is nice cus it looks like the single variable chain rule.

Grant
\href{https://www.khanacademy.org/math/multivariable-calculus/multivariable-derivatives/multivariable-chain-rule/v/multivariable-chain-rule-and-directional-derivatives}{points
out here} that the chain rule is basically a directional derivative in
the direction of \(\vec{v}'(t)\).

Formal treatment of the multivariable chain rule
\href{https://youtu.be/5mMLaK1ByZc}{here}.

    \subsection{Curvature}\label{curvature}

If you imagine you're driving on a road represented by a graph, and the
steering wheel froze up at a given point, you'd have a circle drawn by
the motion at any given point. The \textbf{radius of curvature} is the
radius of that circle. The actual amount of curvature is the reciprocal
of the radius of curvature, \(\kappa = \frac{1}{R}\).

To calculate the curvature, we look at how quickly the unit vectors
tangent to the curve change direction. Specifically, we look at the size
of the derivative of the rate of change of the tangent vectors with
respect to the arc length:

\[\kappa = ||\frac{dT}{ds}||\]

\subsubsection{Computing curvature}\label{computing-curvature}

We are given a function \(\vec{s}(t)\) which defines a curve in the xy
plane.

First, we need to find a unit tangent vector. \(\vec{T}(t)\) is a
vector-valued function which takes in a parameter \(t\) and ouputs a
unit vector which is tangent to the curve at \(\vec{s}(t)\). We can
express the unit tangent vector as a function of time by dividing out
the magnitude, which normalizes to unit vectors:

\[T(t) = \frac{\vec{s}'(t)}{||\vec{s}'(t)||}\]

Next, we need to find \(||\frac{dT}{ds}||\), the derivative of unit
tangent vector with respect to arc length. The typical way to find this
is:

\[\kappa = ||\frac{dT}{ds}|| = \frac{||\frac{dT}{dt}||}{||\frac{d\vec{s}}{dt}||}\]

\(\frac{d\vec{s}}{dt}\) is pretty straightforward to find. It's just the
derivative of the vector-valued function \(\vec{s}\), which is just the
derivative of the two components of \(\vec{s}\).

Now we can solve for the derivative of the unit tangent,
\(\frac{dT}{dt}\), and take its magnitude to get \(||\frac{dT}{dt}||\).
Finally, divide by \(||\vec{s}'(t)||\), which we should have already
calculated in the process of finding \(T\). From this division we have
kappa.

\subsubsection{The direct formula}\label{the-direct-formula}

We can hop ahead to kappa directly with this formula:

\[\kappa = \frac{x'y'' - y'x''}{((x')^2 + (y')^2)^{\frac{3}{2}}}\]

    \subsection{Divergence}\label{divergence}

Imagine a vector field as fluid flow, where particles move along the
vectors over time. If you have a given point, does fluid tend to flow
towards that point, or does it \textbf{diverge} away from it.

Positive divergence means particles flow away from the point on net.
Negative divergence means particles flow towards the point on net. When
divergence is 0, the amount flowing in is balanced with the amount
flowing out.

\subsubsection{Divergence formula}\label{divergence-formula}

We have a vector-valued function,
\(\vec{v}(x,y) = \begin{bmatrix}P(x,y) \\ Q(x,y)\end{bmatrix}\). Then
\(\text{div}\vec{v}(x,y) = \frac{\partial P}{\partial x} + \frac{\partial Q}{\partial y}\).

There's another notation for divergence:

\[\nabla \cdot \vec{v}\]

This makes sense if you think of \(\nabla\) as a vector of partial
derivative operators.

    \subsection{Curl}\label{curl}

Say we have particles moving around a vector field. In a region where
there's counter-clockwise rotation, we say the curl is positive. In a
region where there's clockwise rotation, we say the curl is negative.
When there's no net rotation, we say there is zero curl.

Sometimes it's not obvious from looking at a graph of the vector field
where the curl is. One trick is to consider it like a spinny wheel at
the point you're at. Does it spi

\subsubsection{Formula for 2d Curl}\label{formula-for-2d-curl}

We have \(\vec{v}(x,y) = \begin{bmatrix}P(x,y) \\ Q(x,y)\end{bmatrix}\).

\[\text{curl}(\vec{v}(x,y)) = \frac{\partial Q}{\partial x} - \frac{\partial P}{\partial y}\]

\subsubsection{3-d Curl}\label{d-curl}

With 2d-rotation, it's possible to represent it all with just one
number, which is how quickly it rotates around. Clockwise vs
counterclockwise can be represented by negative vs positive. When we
look at 3d rotation, it's not enough to have just one number. We need
both the axis and the speed of rotation to completely describe the
rotation, in other words a vector.

The rate at which rotation occurs in 3d is called the \textbf{angular
velocity}. The direction describes the \textbf{axis of rotation}. But
which way should the vector point? Since it could go both ways. For this
we have a convention called the right hand rule.

\begin{figure}
\centering
\pandocbounded{\includegraphics[keepaspectratio,alt={image.png}]{images/right-hand-rule.png}}
\caption{image.png}
\end{figure}

We need a 3-dimensional vector to describe the rotation in 3D. The
magnitude of the vector gives the angular velocity, and the direction
gives the axis of rotation.

So how do we compute 3d-curl? With a 3d vector field we have:
\(\vec{v}(x,y,z) = \begin{bmatrix} P(x,y,z) \\ Q(x,y,z) \\ R(x,y,z)\end{bmatrix}\).
What we want to find is the rotation at any given x,y,z coordinate.

{[}I'm stopping part way through here. I can continue on multivariable
later on this section if I decide to{]}

\end{document}
