\documentclass[11pt]{article}
\usepackage{math-notes-preamble}

\title{Physics}

\begin{document}
\maketitle

\section{Position, Velocity, and Accelaration as
Derivatives}\label{position-velocity-and-accelaration-as-derivatives}

Change in Position with respect to time is Velocity. Change in Velocity
with respect to time is Acceleration. As the double derivative,
acceleration is measured in seconds squared, which could be phrased as
``per second per second.''

Speed can be thought of as absolute velocity, not caring which direction
your position is moving in. Positive and negative velocity count towards
it. Speed = \(|v(t)|\)

Position, Velocity, and Acceleration are all Vector quantities. Speed is
scalar.

\section{Displacement vs Distance
Traveled}\label{displacement-vs-distance-traveled}

Good overview on \href{https://youtu.be/Sy_7PkoTCtA}{Khan vid}.

Displacement is overall change in position. Distance Traveled is total
length of the path. Where you start at matters for neither.

Displacement is integral of Velocity. Distance Traveled is integral of
Speed.

Displacement has direction, it's a vector. Distance Traveled lacks
direction, it's a scalar.

So when you're solving for displacement,
\href{https://youtu.be/M7IZClYqdEo}{problems} will require picking your
starting point and adding that to the definite integral from there. When
you're solving for Distance Traveled you wont have the addition and
you'll take the definite integral with respect to abs value.

\section{Force}\label{force}

Force is a vector quantity.

\end{document}
