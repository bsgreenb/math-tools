\documentclass[11pt]{article}
\usepackage{math-notes-preamble}

\title{Sets}

\begin{document}
\maketitle

Note: I wrote this out a while ago, before I did \emph{The Book of
Proof}. I would recommend to go through The Book of Proof for learning
the stuff you see here which was pulled from less organized or complete
sources.

See also: \url{Logic}

A \textbf{Set} is a collection of of objects called elements. Sets can
be finite or infinite.

\[ A = \{1,2,3\} \]

\[ \mathbb{Z}^+ = \{1,2,3,4...\} \]

There is no repitition or order in sets:

\[ \{3,2,1\} = \{1,2,3,3\} \]

Our \textbf{Universe} is the set containing all the elements in our
scope.

\section{Sets of Numbers}\label{sets-of-numbers}

\$ \mathbb{R} \$ The set of real numbers.

\$ \mathbb{Z} \$ The set of integers. The numbers
\{\ldots,-3,-2,-1,0,1,2,3,\}.

\$ \mathbb{N} \$ The set of natural numbers. The numbers
\{1,2,3,4,\ldots\}

\$ \mathbb{C} \$ The set of complex numbers.

\section{Elements and Cardinality}\label{elements-and-cardinality}

\$ \in \$ is an element of

\$ \notin \$ is not an element of

\$ \textbar S\textbar{} \$ number of elements of S
(\textbf{Cardinality})

\section{The empty set}\label{the-empty-set}

\[ \emptyset = \{\} \]

\[ |\emptyset| = 0 \]

\[ |\{\emptyset\}| = 1 \]

\section{Set Builder and Interval
Notation}\label{set-builder-and-interval-notation}

\textbf{Set Builder Notation}: Elements on the left, conditions (or
really predicate) on the right.

\[ \{\frac{m}{n} | m,n\in\mathbb{Z},n \ne 0\} \]

\textbf{Interval Notation}:

\begin{itemize}
\tightlist
\item[$\square$]
  a square bracket when we want to include the end value, or
\item
  ( ) a round bracket when we do Not want to
\end{itemize}

\section{Cartesian Products}\label{cartesian-products}

An \textbf{ordered pair} \$ (a, b) \$ is a set

\[ \{\{a\}, \{a,b\}\} \]

This provides order without sets having built in order. This allows you
to distinguish \((2,1)\) from \((1,2)\).

The \textbf{Cartesian Product} \$ A \times B \$, which is the set:

\[ \{(x,y) | x \in A, y \in B\} \]

Don't confuse these with regular math products, they dont't have the
same associativity and commutativity, via the order mattering. Note that
the cardinality does follow those rules tho..

Cardinality of Cartesian Products

\[ |A| \times |B| = |A \times B| \]

Empty set times anything will get you back the empty set:

\[ \emptyset \times A = \emptyset \]

\section{Subsets}\label{subsets}

If A is a \textbf{Subset} of B, then every element in A must also be in
B. Subset

\[ \{1,2\} \subset \{1,2,3\} \] \[ \{1,2\} \subset \{1,2\} \]

Proper subset means its not the same set, the subset being smaller.
People diverge in their use here
https://math.stackexchange.com/questions/1114920/does-anyone-use-subset-for-proper-subset-anymore

\[ \{1,2\} \subsetneq \{1,2,3\} \]

The empty set is a Subset of everything:

\[ \emptyset \subset A\]

\section{Power Sets}\label{power-sets}

A \textbf{Power Set} of A, is the set containing all possible subsets of
A. We can use \(p(A)\) for this, since I dont wanna use the BB notation
for dis.

 
\begin{align*}
& A = \{a,b\} \\
& p(A) = \{\{a,b\}, \{a\}, \{b\}, \emptyset \} \\
\end{align*}

The size/cardinality of the powerset of a set is a power of 2 of that
set's cardinality.

\[ |p(A)| = 2^{|A|} \] \[ p(\emptyset) = \{\emptyset\} \]

\section{Set Operations}\label{set-operations}

\textbf{Complement}: \$ A' = \bar\{A\} = \{ x \textbar{} x \notin A \}
\$.

\textbf{Intersection}: \$ A \cap B = \{ x \textbar{} x \in A, x \in B \}
\$

\textbf{Union}: \$ A \cup B = \{ x \textbar{} x \in A \lor x \in B \} \$

\textbf{Difference}: \$ A - B = A \setminus B = \{ x \textbar{} x
\in A,x \notin B \} \$

\textbf{Symmetric Difference} or \textbf{Disjoint Union}: \$ A \oplus B
= A \triangle B = A \{ x \textbar{} x \in A \oplus x \in B \} \$

Note I wrote out both notations I'm seein out there for Complement,
Difference, and Symmetric Difference.

\section{Set Laws}\label{set-laws}

Note that these rules have corresponding Logic Laws. I believe this is
because they're both boolean algebras.

https://web.uvic.ca/\textasciitilde gmacgill/LFNotes/LawsOfSetTheory.pdf

\section{Indexed Sets}\label{indexed-sets}

You can iterate over Union and Intersection with this notation:

\textbf{Union of Indexed Set} \[ 
\bigcup_{i=0}^{n} F_{i} = {A}_{0}\cup{A}_{1}\cup{A}_{2}\cup\dots\cup{A}_n 
\]

\textbf{Intersection of Indexed Set} \[ 
\bigcap_{i=0}^{n} F_{i} = {A}_{0}\cap{A}_{1}\cap{A}_{2}\cap\dots\cap{A}_n 
\]

    \section{Partition}\label{partition}

A \textbf{partition} of a set \(A\) is a set of non-empty subsets of
\(A\), such that the union of all the subsets equals \(A\), and the
intersection of any two subsets is \(\emptyset\).

    \section{See Also}\label{see-also}

\href{RelationsFunctions.ipynb}{Relations/Functions} which are based
like everything else on sets.

\end{document}
