\documentclass[11pt]{article}
\usepackage{math-notes-preamble}

\title{CalcVectorsParametricPolar}

\begin{document}
\maketitle

\section{Parametric Equations}\label{parametric-equations}

Time (\(t\)) tends to be the parameter used, but it could be something
else.

\subsection{Defining and differentiating parametric
equations}\label{defining-and-differentiating-parametric-equations}

First derivative:

\[\frac{dy}{dx} = \frac{\frac{dy}{dt}}{\frac{dx}{dt}}\]

Second derivative:

\[ \frac{d^2y}{dx^2} = \frac{d}{dx}\left(\frac{dy}{dx}\right) = \frac{\frac{d\left(\frac{dy}{dx}\right)}{dt}}{\frac{dx}{dt}}\]

\subsection{Parametric Curve Arc
Length}\label{parametric-curve-arc-length}

Let's start with the magic differential approach taken on the
\href{https://www.khanacademy.org/math/ap-calculus-bc/bc-advanced-functions-new/bc-9-3/v/parametric-curve-arc-length?modal=1}{Khan
Vid}:

\begin{figure}
\centering
\pandocbounded{\includegraphics[keepaspectratio,alt={image.png}]{images/parametric-arc-length.png}}
\caption{image.png}
\end{figure}

Using the pythagorean theorem here we get:

\[ 
L = \int_a^b ds = \int_a^b \sqrt{(dx)^2 + (dy)^2} \\
\int_a^b \sqrt{(\frac{dx}{dt}dt)^2 + (\frac{dy}{dt}dt)^2} 
\]

Then factoring out the \(dt\)'s magically:

\[
\int_a^b \sqrt{(dt)^2(\frac{dx}{dt})^2 + (\frac{dy}{dt})^2} \\
\int_a^b \sqrt{(\frac{dx}{dt})^2 + (\frac{dy}{dt})^2}dt
\]

But the more rigorous proof is on
\href{https://tutorial.math.lamar.edu/classes/calcii/ParaArcLength.aspx}{Paul's
Math Notes}. It does some funky stuff with absolute value I'm not
completely groking.

    \section{Polar Curves/Equations}\label{polar-curvesequations}

Good \href{https://brilliant.org/wiki/polar-curves/}{Overview on
Brilliant}.

\begin{quote}
A polar curve is a shape constructed using the polar coordinate system.
Polar curves are defined by points that are a variable distance from the
origin (the pole) depending on the angle measured off the positive
x-axis. Polar curves can describe familiar Cartesian shapes such as
ellipses as well as some unfamiliar shapes such as cardioids and
lemniscates.
\end{quote}

One advantage of using polar equations is that certain relations that
are not functions in Cartesian form can be expressed as functions in
polar form.

\subsection{Derivatives of Polar
Curves}\label{derivatives-of-polar-curves}

Derivatives of Polar Curves To find \(\frac{d y}{d x}\) given that
\(x=r \cos \theta\) and \(y=r \sin \theta\), we can find
\(\frac{d y}{d \theta}\) and \(\frac{d x}{d \theta}\). Upon dividing the
former by the latter, we get \(\frac{d y}{d x}\).

For a polar curve, \[
\frac{d y}{d x}=\frac{\frac{d r}{d \theta} \sin \theta+r \cos \theta}{\frac{d r}{d \theta} \cos \theta-r \sin \theta}
\] Important facts to note: - When \(\frac{d y}{d \theta}=0\), we have
horizontal tangent lines. - When \(\frac{d x}{d \theta}=0\), we have
vertical tangent lines.

\subsection{Areas Bound by Polar
Curves}\label{areas-bound-by-polar-curves}

Whereas with a cartestian graph we'd use rectangles to get an integral,
with a polar graph it's more taking a bunch of rotating pie pieces.
Infinitely thin pie slices. Area of these sectors given by:

\[\frac{\theta}{2\pi}\pi r^2 = \frac{\theta}{2}r^2 = \frac{1}{2}r^2\theta\]

\(\theta\) -\textgreater{} \(d\theta\) as slice gets infinitely small..
So we can calculate the slices over a given angular interval as:

\[ \int_\alpha^\beta \frac{1}{2}r^2~d\theta \]

\begin{figure}
\centering
\pandocbounded{\includegraphics[keepaspectratio,alt={image.png}]{CalcVectorsParametricPolar_files/928f7302-b688-4ca8-ae19-0522aa577d21.png}}
\caption{image.png}
\end{figure}

So if you're calculating the area between polar curves, the key thing
will be to identity where those curves intersect, and use that to
determine your integrals.

Be on the lookout for symmetries when you're looking at areas between
curves.

    \section{Vectors from Calc}\label{vectors-from-calc}

{[}MAKES SENSE to integrate this in line with below sections from Linear
Alg{]}

Using
\href{https://www.khanacademy.org/math/get-ready-for-ap-calc/xa350bf684c056c5c:get-ready-for-parametric-polar-vector}{Khan}
to get the basics here just as part of doing AP Calc. But will expand
this once taking on Linear Algebra.

A \textbf{Vector} has a \textbf{magnitude/size} and a
\textbf{direction/angle}. A \textbf{Scalar} only has the former.

A vector is not defined by it's starting point, just its magnitude and
direction. It could be dropped anywhere on a graph.

Vectors can be split up into Parametric Equations.
\href{https://math.stackexchange.com/a/1940372/49487}{0 upvote MSE
Question} which hopefully is correct:

\begin{quote}
parametric equation with only one parameter, it's the same thing. The
parametric equations are the components of the vector function.
\end{quote}

\href{https://en.wikipedia.org/wiki/Euclidean_vector}{Euclidean Vectors}
are all I dealt with in Calc. When taking on Linear, will move beyond
these. \href{https://en.wikipedia.org/wiki/Position_(geometry)}{Position
Vectors} are Euclidean Vectors that start at the origin at point to
something in Euclidean space.

\href{https://en.wikipedia.org/wiki/Vector_space}{Vector Space} {[}Write
formal definition of vector (Spaces) as part of linear alg{]}

\subsection{Norm / Magnitude}\label{norm-magnitude}

Vector magnitude from components: The magnitude of \((a, b)\) is
\(\|(a, b)\|=\sqrt{a^{2}+b^{2}}\).

\subsection{Direction of Vectors}\label{direction-of-vectors}

Arctan has a limited range, so if you want to get an angle from the
vector components. In Quadrants II and III, you'll wanna add 180. In
Quadrant IV, you'll get a negative angle which you can add 360 to to get
a positive one.

If you're using \href{https://en.wikipedia.org/wiki/Atan2}{atan2}, it
works a bit different than that.

\subsection{Vector components from magnitude and
direction}\label{vector-components-from-magnitude-and-direction}

The components of a vector with magnitude \(\|\vec{u}\|\) and direction
\(\theta\) are
\((\|\vec{u}\| \cos (\theta),\|\vec{u}\| \sin (\theta))\).

\subsection{Unit Vectors}\label{unit-vectors}

\href{https://en.wikipedia.org/wiki/Unit_vector}{Unit vector} is a
vector of length 1.

\href{https://math.stackexchange.com/questions/965477/unit-vector-symbols-names}{Notation
varies} for them, but Khan seems to prefer \(\hat i, \hat j\).

\subsection{Multiplying and Dividing
Vectors}\label{multiplying-and-dividing-vectors}

\textbf{Scalar Multiplication} is scalar times a vector. Pretty
straightforward since it just multiplies the components.

\subsection{Vector valued functions}\label{vector-valued-functions}

Typical notation for a \textbf{Vector Function} aka \textbf{Vector
Valued Function} is either arrow over like \(\vec{v}\) (I prefer, cus I
write it same with pencil) or else bold it.

\(\vec{r}(t) = x(t)\hat i + y(t)\hat j\)

From this definition, and the limit definition of a derivative, we see
that we can get the derivative of vector functions.

\subsubsection{Vector Valued Function
Differentiation}\label{vector-valued-function-differentiation}

The derivative of a vector is itself a vector.

\(\vec{r}'(t) = x'(t)\hat i + y'(t)\hat j\)

\end{document}
