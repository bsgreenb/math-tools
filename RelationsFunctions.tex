\documentclass[11pt]{article}
\usepackage{math-notes-preamble}

\title{RelationsFunctions}

\begin{document}
\maketitle

\section{Relations}\label{relations}

A \textbf{relation} \(R\) on a set \(X\) is a subset of \$ X \times X
\$. (Note: \textasciitilde{} can also be used for relation in place of
R).

A \textbf{binary relation} \(R\) from \(A\) to \(B\) is a subset of \$ A
\times B \$, which is a collection of ordered pairs \((a, b)\) where
\(a\) comes from \(A\) and \(b\) comes from \(B\). We can write \(aRb\)
to mean \(a\) is related to \(b\).

Example definition of greater than relation on the set of real numbers.

\[ R : \{(x,y) \in \mathbb{Z} | x > y\} \]

\subsection{Reflexive, Symmetric, and Transitive
Relations}\label{reflexive-symmetric-and-transitive-relations}

\textbf{Reflexive}: \(\forall x[xRx]\)

\textbf{Symmetric}: \(\forall(x,y)[xRy \rightarrow yRx]\)

\textbf{Transitive}: \$\forall(x,y,z){[}xRy \land yRZ \rightarrow xRZ{]}
\$

Note on syntax used here: It should be fine to use one quantifier on
multiple variables
https://math.stackexchange.com/questions/4323172/is-there-a-difference-between-using-two-vs-one-universal-quantifier-for-two-vari

Operators can be expressed in terms of these types of relations. For
example \(\leq\) is reflexive and transitive, but not symmetric. \(=\)
is all 3, making it and equivalence relation.

\textbf{Equivalence relation} means it's Reflexive, Symmetric, and
Transitive. Suppose \(R\) is an equivalence relation on set \(X\). Then,
we can represent the \textbf{equivalence classes}
\(\forall x \in X [x] = \{y|xRy\}\)

Basically, \([x]\) is set of all \(y\) related to \(x\). We can express
an equivalence relation as \([x] = [y] \lor [x] \cap [y] = \emptyset\).
Let's get more concrete for a sec:

\begin{align*}
& X = \{1,2,3\} \\
& R = \{(1,1),(2,2),(3,3),(2,3),(3,2) \} \\
& [1] = \{1\}, [2] = [3] = \{2, 3\} \quad\text{We get these equivalence classes}\\
\end{align*}

Moving on, \textbf{Antisymmetry} is defined as: \$ (aRb \land bRa)
\rightarrow a = b\$. Example operators that fall under this definition:
\(\leq, \geq, \subset\).

Being a \textbf{Partial Order} means a relation is Anti-symmetric,
Reflexive, and Transitive. The previously mentioned operators can be
described as such, as could alphabetical ordering.

    \section{Functions}\label{functions}

{[}Note: This writeup could definitely be improved{]}

\[ f : X \rightarrow Y \]

A \textbf{function} \(f\) maps \(X\) to \(Y\). \(X\) is the
\textbf{Domain}, \(Y\) is the \textbf{Codomain}.
\textbf{Range}/\textbf{Image} is a subset of Codomain, and is set of all
values of \(y = f(x)\). The set of all \(x\) in the Domain (\(x \in X\))
is called the \textbf{Pre-Image}.

\href{https://en.wikipedia.org/wiki/Function_(mathematics)}{Wiki} seems
better on this than the stuff I was reading.

The key requirement of a function is that there is never more than one
value of f(x) for a given x. If this rule is broken, what you're dealing
with is a relation.

You can also think of a function as the cartesian product of the domain
and the co-domain, indexed by the domain.

Functions have the requirement that there's one associated image for
each x in the domain. We can call the domain initially under
consideration the \textbf{Domain of Definition} (another called it
\textbf{Pre-Domain}). \textbf{Total Functions} have the whole Domain of
Definition as their Domain -- every x has an image f(x).
\href{https://en.wikipedia.org/wiki/Partial_function}{Partial Functions}
have values that aren't fully defined (don't have an image f(x)). You'll
typically just here a Partial Function called a Function.. because when
we look at something like with no 0 allowed in denominator, we're
basically dealing with a Partial Function and act accordingly. It's
basically like in coding where you either have f(x) in Y or else f(x) is
undefined.

You can technically return multiple values from a function if you return
them as a Set. Then you got a
\href{https://en.wikipedia.org/wiki/Multivalued_function}{Multivalued
function}
\begin{align*}
& f(x) = x^2, x \in \mathbb{Z} \\
& f : X \rightarrow \mathbb{Z^+} \\
& \mathrm{range}(f) = \{x^2 \in \mathbb{Z^+} | \sqrt{x} \in \mathbb{Z} \}
\end{align*}

{[}should have the image here{]}

Good overview of domain/codomain/range by
\href{https://www.mathsisfun.com/sets/domain-range-codomain.html}{MathIsFun}.
Key point, is the codomain is kind of a bounding guess \textgreater{}
sometimes we don't know the exact range (because the function may be
complicated or not fully known), but we know the set it lies in (such as
integers or reals). So we define the codomain and the range is a subset
of that.

\subsection{Characteristic Function}\label{characteristic-function}

\[
f(x) =
\begin{cases}
0 ~&\text{ if }~ x \notin A \\
1 ~&\text{ if }~ x \in A
\end{cases}
\]

Acts like a truth table for set membership.

\subsection{Injective, Surjective, and Bijective
Functions}\label{injective-surjective-and-bijective-functions}

Good
\href{https://en.wikipedia.org/wiki/Function_(mathematics)\#Injective,_surjective_and_bijective_functions}{coverage
on wiki}.

Let's describe the case of

\[ f : X \rightarrow Y \]

\textbf{Injective} - one-to-one. No repeat y values. f has a
\textbf{left inverse}

If \(x_1 \neq x_2\), then \$f(x\_1) \neq f(x\_2) \$.

\textbf{Surjective} - onto. every value of Y (the co-domain) is covered
by a corresponding input x. Range = codomain in this case. f has a
\textbf{right inverse}

\$ \forall y \in Y{[}\exists x \in X{[}f(x) = y{]}{]} \$

\textbf{Bijective} - injective and surjective (see above). \(|X| = |Y|\)
for these. f has an \textbf{inverse}

\subsection{Composite Functions}\label{composite-functions}

\begin{align*}
& f: X \rightarrow Y \\
& g: Y \rightarrow Z \\
& g \circ f \rightarrow Z  \\
& (g \circ f)(x) = g(f(x))
\end{align*}

Composite Functions are a more specific case of
\href{https://en.wikipedia.org/wiki/Composition_of_relations}{Composite
Relations}.

Also one thing I'm seeing is that the members of a set are implicitly
the lower case of it, like element \(x\) in set \(X\).

Injectivity and Surjectivity of Composite Functions:

\begin{itemize}
\tightlist
\item
  If \(f\) and \(g\) are injective, then \(g \circ f\) is injective.
\item
  If \(f\) and \(g\) are surjective, then \(g \circ f\) is surjective.
\item
  If \(f\) and \(g\) are bijective, then \(g \circ f\) is bijective.
\end{itemize}

\subsection{Inverse Functions}\label{inverse-functions}

The \textbf{inverse} of \(f : A \rightarrow B\) is a function
\(f^{-1}: B \rightarrow A\) such that

\[
x=f^{-1}(y) \text { if and only if } y=f(x)
\]

Can think of an inverse as undoing a function. \(1/x\) its own inverse,
\(ln(x)\) and \(e^x\) each others inverses (over a restricted domain).

A function \(f : A \rightarrow B\) has an inverse if and only if it is
bijective. In that case, the inverse \(f^{-1}\) is a bijective function
from \(B\) to \(A\).

Functions that are their own inverses are called \textbf{involution}s.

\subsection{Image and Pre-Image}\label{image-and-pre-image}

Suppose \(f: A \rightarrow B\) is a function. 1. If \(X \subseteq A\),
the image of \(X\) is the set \(f(X)=\{f(x): x \in X\} \subseteq B\). 2.
If \(Y \subseteq B\), the preimage of \(Y\) is the set
\(f^{-1}(Y)=\{x \in A: f(x) \in Y\} \subseteq A\).

\subsection{Periodicity}\label{periodicity}

A function f is said to be \textbf{periodic} of period a if there is a
number a, called the period of f, such that f(x) = f(x + na) for every x
∈ and n ∈ .''

    A function \(f\) is said to be \textbf{periodic} of period \(a\) if
there is a number \(a\), called the \textbf{period} of \(f\), such that
\(f(x) = f(x + na)\) for every \(x \in \mathbb{R}\) and
\(n \in \mathbb{Z}\).

\end{document}
